%\documentclass[12pt,letterpaper]{article}
\documentclass[12pt]{amsart}

\usepackage[left=1in, right=1in, top=.5in, bottom=.5in]{geometry}
\usepackage{latexsym,amssymb,amsmath,amsthm,amsopn,verbatim, mathpazo, graphicx, lmodern}
%\usepackage{latexsym,amssymb,amsmath,amsthm,amsopn,verbatim, mathpazo, graphicx}
\newcommand{\vs}{\vskip.5cm}
\newcommand{\hs}{\hskip1cm}
\newcommand{\ds}{\displaystyle}
\setlength{\parindent}{0pt}

\usepackage{hyperref}  % links, urls
\urlstyle{same}

\usepackage[T1]{fontenc}
\usepackage{libertine}
\renewcommand*\familydefault{\sfdefault}  %% Only if the base font of the document is to be sans serif


\newtheorem{theorem}{Theorem}[section]
\newtheorem{corollary}{Corollary}[theorem]
\newtheorem{lemma}[theorem]{Lemma}
\newtheorem{definition}[theorem]{Definition}
\newtheorem{example}[theorem]{Example}

\newcommand\indep{\protect\mathpalette{\protect\independenT}{\perp}}
\def\independenT#1#2{\mathrel{\rlap{$#1#2$}\mkern2mu{#1#2}}}

\usepackage{fancyhdr}
\pagestyle{fancy}
\fancypagestyle{plain}{}
%\fancyhead{} % clear all header fields
%\fancyhf{}
\rhead{BSTA 550 Ch 10}   % from fancyhdr package
\renewcommand{\headrulewidth}{1pt}
\renewcommand{\footrulewidth}{0pt}



\begin{document}


%----------------------------------------------------------------------------
\setcounter{section}{10}
{\huge  
\section*{Chapter 10 \newline Expected Values of Discrete r.v.'s}
}
%----------------------------------------------------------------------------

%----------------------------------------------------------------------------
{\large 
%----------------------------------------------------------------------------

%----------------------------------------------------------------------------
%\vspace{18cm}
%\hrule
\vspace{1cm}


%----------------------------------------------------------------------------
\begin{example}\label{Die expected outcome}
Suppose you roll a fair 6-sided die. \newline
What value do you expect to get?

\end{example}

\textbf{Solution:}


%----------------------------------------------------------------------------
\vspace{8cm}
\hrule
\vspace{.5cm}

%----------------------------------------------------------------------------
\begin{example}\label{Die expected outcome}
Suppose the die is 6-sided, but not fair. \newline
What value do you expect to get on a roll?

\end{example}

\textbf{Solution:}





%****************************************************************************************
\newpage

\textbf{Remark:} Expected vs. actual outcomes%\newline

%----------------------------------------------------------------------------
\vspace{4cm}
\hrule
\vspace{.5cm}

%----------------------------------------------------------------------------


%----------------------------------------------------------------------------
\begin{definition}[Expected value]
The \textbf{expected value} of a discrete r.v. $X$ that takes on values $x_1,  x_2, \ldots, x_n$ is
$$
\mathbb{E}[X] = \sum_{i=1}^n x_ip_X(x_i).
$$ 
\end{definition}

%----------------------------------------------------------------------------
\vspace{.5cm}


%----------------------------------------------------------------------------
\textbf{Remarks:} %\newline
\begin{itemize}
\item 
The definition holds if the r.v. $X$ takes on countably infinitely many values $x_1,  x_2, \ldots$, as well:
$$
\mathbb{E}[X] = \sum_{i=1}^{\infty} x_ip_X(x_i).
$$
%\item The (finite) expected value exists only if 
%$$
%\sum_{i=1}^{\infty} |x_i|p_X(x_i) < \infty
%$$
\item Another way to define the expected value of a discrete r.v. is to do so at the $\omega$ level, where the $\omega$'s are outcomes in the sample space:
	\begin{itemize}
	\item Suppose $\omega_1, \omega_2, \ldots, \omega_n$ are the possible outcomes of a random phenomenon. If outcome $\omega_i$ causes the r.v. X to take on value $x_i$ (meaning $X(\omega_i)=x_i$), then  
	$$
		\mathbb{E}[X] = \sum_{i=1}^{\infty} x_i\mathbb{P}(\{\omega_i\}).
	$$
	\end{itemize}
\end{itemize}

 

%%----------------------------------------------------------------------------
%\vspace{.5cm}
%\hrule
%\vspace{.5cm}

%----------------------------------------------------------------------------

%****************************************************************************************
\newpage

%----------------------------------------------------------------------------
\begin{example}\label{Bernoulli_E}
Suppose
$$
X = \left\{
        \begin{array}{ll}
            1 & \quad \mathrm{with\ probability}\ p \quad\mathrm{(success)}\\
            0 & \quad \mathrm{with\ probability}\ 1-p \quad\mathrm{(failure)}
        \end{array}
    \right.
$$
Find the expected value of $X$.
\end{example}

\textbf{Solution:}

%----------------------------------------------------------------------------
\vspace{8cm}
\hrule
\vspace{.5cm}

%----------------------------------------------------------------------------

%----------------------------------------------------------------------------
\begin{example}
Suppose
$$
X = \left\{
        \begin{array}{ll}
            1 & \quad \mathrm{with\ probability}\ p \\
            -1 & \quad \mathrm{with\ probability}\ 1-p 
        \end{array}
    \right.
$$
Find the expected value of $X$.
\end{example}

\textbf{Solution:}



%****************************************************************************************
\newpage

%----------------------------------------------------------------------------
\begin{example}\label{Bullseye}
Suppose I throw darts at a dartboard until I hit the bullseye, and that my probability of hitting the bullseye is $p$. Suppose further that all of my throws are independent, and that the probability of a bullseye never changes, no matter how many times I throw a dart. \newline
How many times should I expect to have to throw the dart until I hit the bullseye?

\end{example}

\textbf{Solution:}


%----------------------------------------------------------------------------



%****************************************************************************************
\newpage

%----------------------------------------------------------------------------
\begin{example}\label{GhostUniformTrickTreat}
A ghost is trick-or-treating. It comes to a house where it is known that there are 30 candies in the bag and only one is a watermelon Jolly Rancher, which is the ghost's favorite. The ghost takes pieces of candy without replacement until it gets the watermelon Jolly Rancher. \newline
How many pieces of candy do we expect the ghost to take?
\end{example}

\textbf{Solution:}


%----------------------------------------------------------------------------


%****************************************************************************************
\newpage

%----------------------------------------------------------------------------
\textbf{Remark:} Both examples are repeated random processes. They are fundamentally different though: \newline
\begin{itemize}
\item The bullseye example (\ref{Bullseye}) is "\textit{with replacement}" since the probability of success remains constant.
\item The ghost trick-or-treating example (\ref{GhostUniformTrickTreat}) is \textit{without replacement}, and thus the probability of success changes with each trial.
\end{itemize}

 


%----------------------------------------------------------------------------
}  % end large font
%----------------------------------------------------------------------------



\end{document}

