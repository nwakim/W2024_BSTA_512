%\documentclass[12pt,letterpaper]{article}
\documentclass[12pt]{amsart}

\usepackage[left=1in, right=1in, top=.5in, bottom=.5in]{geometry}
\usepackage{latexsym,amssymb,amsmath,amsthm,amsopn,verbatim, mathpazo, graphicx, lmodern}
%\usepackage{latexsym,amssymb,amsmath,amsthm,amsopn,verbatim, mathpazo, graphicx}
\newcommand{\vs}{\vskip.5cm}
\newcommand{\hs}{\hskip1cm}
\newcommand{\ds}{\displaystyle}
\setlength{\parindent}{0pt}

\usepackage{hyperref}  % links, urls
\urlstyle{same}

\usepackage[T1]{fontenc}
\usepackage{libertine}
\renewcommand*\familydefault{\sfdefault}  %% Only if the base font of the document is to be sans serif


\newtheorem{theorem}{Theorem}[section]
\newtheorem{corollary}{Corollary}[theorem]
\newtheorem{lemma}[theorem]{Lemma}
\newtheorem{definition}[theorem]{Definition}
\newtheorem{example}[theorem]{Example}

\newcommand\indep{\protect\mathpalette{\protect\independenT}{\perp}}
\def\independenT#1#2{\mathrel{\rlap{$#1#2$}\mkern2mu{#1#2}}}

\usepackage{fancyhdr}
\pagestyle{fancy}
\fancypagestyle{plain}{}
%\fancyhead{} % clear all header fields
%\fancyhf{}
\rhead{BSTA 550 Ch 12}   % from fancyhdr package
\renewcommand{\headrulewidth}{1pt}
\renewcommand{\footrulewidth}{0pt}



\begin{document}


%----------------------------------------------------------------------------
\setcounter{section}{12}
{\huge  
\section*{Chapter 12: Variance of Discrete r.v.'s - \newline
or, Expected Values of Functions of r.v.'s}
}
%----------------------------------------------------------------------------

%----------------------------------------------------------------------------
{\large 
%----------------------------------------------------------------------------

%----------------------------------------------------------------------------
%\vspace{18cm}
%\hrule
%\vspace{1cm}

\subsection{Expected Values of Functions of r.v.'s}\ \newline

\vspace{.5cm}

\textbf{Question:} What is $\mathbb{E}[g(X)]$ for a function $g$ and discrete r.v. $X$?

%----------------------------------------------------------------------------
\vspace{.5cm}
\hrule
\vspace{.5cm}

%----------------------------------------------------------------------------
\begin{example}
Let $g(x) = ax+b$, for real-valued constants $a$ and $b$. \newline
What is $\mathbb{E}[g(X)]$?
\end{example}

\textbf{Solution:}


%----------------------------------------------------------------------------
\vspace{8cm}
\hrule
\vspace{.5cm}




%----------------------------------------------------------------------------
\begin{definition}[Expected value of a function of a r.v.]\label{EfcnX}\ \newline
For any function $g$ and discrete r.v. $X$, the expected value of $g(X)$ is
$$
\mathbb{E}[g(X)] = \sum_{\{all\ x\}}\ g(x) p_X(x).
$$ 
\end{definition}






%****************************************************************************************
\newpage


%----------------------------------------------------------------------------
\begin{example}\label{Draw2Cards_gX}
Suppose you draw 2 cards from a standard deck of cards \textit{with} replacement. %\newline
Let $X$ be the number of hearts you draw. 

\begin{enumerate}
\item Find $\mathbb{E}[X^2]$.

\textbf{Solution:}

\vspace{8cm}
\item Find $\mathbb{E}\big[\big(X-\frac{1}{2}\big)^2\big]$.

\textbf{Solution:}
\vspace{8cm}

\end{enumerate}
\end{example}



%****************************************************************************************
\newpage

\subsection{Variance of a r.v.}%\ \newline
%----------------------------------------------------------------------------
\begin{definition}[Variance of a r.v.]\label{DefVar}\ \newline
The variance of a r.v. $X$, with (finite) expected value $\mu_X=\mathbb{E}[X]$ is
$$
\sigma_X^2=Var(X)=\mathbb{E}[(X-\mu_X)^2] = \mathbb{E}[(X-\mathbb{E}[X])^2].
$$ 
\end{definition}

%----------------------------------------------------------------------------
\vspace{3cm}
%----------------------------------------------------------------------------
\begin{definition}[Standard deviation of a r.v.]\label{DefSD}\ \newline
The standard deviation of a r.v. $X$ is
$$
\sigma_X = SD(X) = \sqrt{\sigma_X^2}=\sqrt{Var(X)}.
$$ 
\end{definition}
%----------------------------------------------------------------------------
\vspace{.5cm}
\hrule
\vspace{.5cm}


\textbf{Questions:} \newline
Why do we square the difference in the variance definition? $(X-\mu_X)^2$
\begin{itemize}
\item Why not define the measure of spread as $\mathbb{E}[X-\mu_X] = \mathbb{E}[X-\mathbb{E}[X]]$?

\vspace{9cm}
\item  Why not use $\mathbb{E}[|X-\mu_X|]$?

\end{itemize}


%****************************************************************************************
\newpage



%----------------------------------------------------------------------------
\begin{lemma}["Computation formula" for Variance]\label{DefVar_Comp}\ \newline
The variance of a r.v. $X$, can be computed as
$$
\sigma_X^2=Var(X)=\mathbb{E}[X^2]-\mu_X^2 = \mathbb{E}[X^2] - (\mathbb{E}[X])^2.
$$ 
\end{lemma}

%----------------------------------------------------------------------------
\begin{proof}
\end{proof}


%----------------------------------------------------------------------------
%\vspace{5cm}


%****************************************************************************************
\newpage

\subsection{Some Important Variance and Expected Values Results}


%----------------------------------------------------------------------------
\begin{lemma}\ \newline
For a r.v. $X$ and constants $a$ and $b$,
$$Var(aX+b) = a^2Var(X).$$
\end{lemma}

\begin{proof} See homework.
\end{proof}

%----------------------------------------------------------------------------
\vspace{.5cm}
\hrule

\subsubsection{Important Results for Independent r.v.'s}

%----------------------------------------------------------------------------
\begin{theorem}\label{E_indep_gXhY}\ \newline
For independent r.v.'s $X$ and $Y$, and functions $g$ and $h$,
$$
\mathbb{E}[g(X)h(Y)] = \mathbb{E}[g(X)]\mathbb{E}[h(Y)].
$$ \end{theorem}

%----------------------------------------------------------------------------
\vspace{.1cm}
%----------------------------------------------------------------------------
\begin{corollary}\ \newline
For independent r.v.'s $X$ and $Y$,
$$
\mathbb{E}[XY] = \mathbb{E}[X]\mathbb{E}[Y].
$$ \end{corollary}

%----------------------------------------------------------------------------
\vspace{.1cm}
%----------------------------------------------------------------------------
\begin{proof}[Proof of Theorem \ref{E_indep_gXhY}] 
\end{proof}




%****************************************************************************************
\newpage

%----------------------------------------------------------------------------
\begin{theorem}[Variance of sum of independent discrete r.v.'s]\label{ThmVarSum}\ \newline
For independent discrete r.v.'s $X_i$ and constants $a_i$, $i=1,2,\dots, n$, 
$$
Var\Big(\sum_{i=1}^n a_iX_i\Big) = \sum_{i=1}^n a_i^2Var(X_i).
$$ 
\end{theorem}

%----------------------------------------------------------------------------
\vspace{.1cm}
%----------------------------------------------------------------------------
\begin{corollary}\ \newline
For independent discrete r.v.'s $X_i$, $i=1,2,\dots, n$, 
$$
Var\Big(\sum_{i=1}^n X_i\Big) = \sum_{i=1}^n Var(X_i).
$$ 
\end{corollary}


%----------------------------------------------------------------------------
\vspace{.1cm}
%----------------------------------------------------------------------------
\begin{corollary}\ \newline
For independent identically (i.i.d.) discrete r.v.'s $X_i$, $i=1,2,\dots, n$, 
$$
Var\Big(\sum_{i=1}^n X_i\Big) = n Var(X_1).
$$ 
\end{corollary}

%----------------------------------------------------------------------------
\vspace{.1cm}
%----------------------------------------------------------------------------
\begin{proof}[Proof to Theorem \ref{ThmVarSum}]
\end{proof}



%****************************************************************************************
\newpage

 
%----------------------------------------------------------------------------
\begin{example}\label{LandportHotels}
A tour group is planning a visit to the city of Landport and needs to book 30 hotel rooms. 
The average price of a room is \$200 \textbf{with standard deviation \$10}. In addition, there is a 10\% tourism tax for each room.
\newline
What is the \textbf{standard deviation} of the cost for the 30 hotel rooms?
\end{example}

\textbf{Solution:}



%----------------------------------------------------------------------------
}  % end large font
%----------------------------------------------------------------------------



\end{document}

