%\documentclass[12pt,letterpaper]{article}
\documentclass[12pt]{amsart}

\usepackage[left=1in, right=1in, top=.5in, bottom=.5in]{geometry}
\usepackage{latexsym,amssymb,amsmath,amsthm,amsopn,verbatim, mathpazo, graphicx, lmodern}
%\usepackage{latexsym,amssymb,amsmath,amsthm,amsopn,verbatim, mathpazo, graphicx}
\newcommand{\vs}{\vskip.5cm}
\newcommand{\hs}{\hskip1cm}
\newcommand{\ds}{\displaystyle}
\setlength{\parindent}{0pt}

\usepackage{hyperref}  % links, urls
\urlstyle{same}

\usepackage[T1]{fontenc}
\usepackage{libertine}
\renewcommand*\familydefault{\sfdefault}  %% Only if the base font of the document is to be sans serif


\newtheorem{theorem}{Theorem}[section]
\newtheorem{corollary}{Corollary}[theorem]
\newtheorem{lemma}[theorem]{Lemma}
\newtheorem{definition}[theorem]{Definition}
\newtheorem{example}[theorem]{Example}

\newcommand\indep{\protect\mathpalette{\protect\independenT}{\perp}}
\def\independenT#1#2{\mathrel{\rlap{$#1#2$}\mkern2mu{#1#2}}}

\usepackage{fancyhdr}
\pagestyle{fancy}
\fancypagestyle{plain}{}
%\fancyhead{} % clear all header fields
%\fancyhf{}
\rhead{BSTA 550 Ch 19}   % from fancyhdr package
\renewcommand{\headrulewidth}{1pt}
\renewcommand{\footrulewidth}{0pt}



\begin{document}


%----------------------------------------------------------------------------
\setcounter{section}{19}
{\huge  
\section*{Chapter 19: Hypergeometric r.v.'s}
}
%----------------------------------------------------------------------------

%----------------------------------------------------------------------------
{\large 
%----------------------------------------------------------------------------

%----------------------------------------------------------------------------
%\vspace{18cm}
%\hrule
%\vspace{1cm}



\vspace{.5cm}

\textbf{Scenario:} There are a fixed number of successes and failures (which are known in advance), from which we make $n$ draws without replacement. We are counting the number of successes from the $n$ trials.

%----------------------------------------------------------------------------
\vspace{.5cm}
\hrule
\vspace{.5cm}


%----------------------------------------------------------------------------
\begin{example}\ %\newline
A wildlife biologist is using mark-recapture to research a wolf population. Suppose a specific study region is known to have 24 wolves,  of which 11 have already been tagged. If 5 wolves are randomly captured, 
what is the probability that 3 of them have already been tagged?

%\begin{itemize}
%\item Suppose a specific study region is known to have 24 wolves, 
%\item of which 11 have already been tagged. 
%\item If 5 wolves are randomly captured, 
%\item what is the probability that 3 of them have already been tagged?
%\end{itemize}

\textbf{Solution:}

\end{example}
%----------------------------------------------------------------------------
\vspace{4cm}
\hrule
\vspace{.5cm}



\textbf{Properties of Hypergeometric r.v.'s}

\begin{itemize}
\item There is a finite population of $N$ items. 
\item Each item in the population is either a success or a failure, and there are $M$ successes total.
\item We randomly select (sample) $n$ items from the population. 
\end{itemize}

%****************************************************************************************
\newpage



\textbf{Hypergeometric vs. Binomial r.v.'s}
%----------------------------------------------------------------------------
\vspace{.5cm}

Suppose a hypergeometric r.v. $X$ has the following properties:
\begin{itemize}
\item the population size $N$ is really big,
\item the number of successes $M$ in the population is relatively large,
	\begin{itemize}
	\item $\frac{M}{N}$ shouldn't be close to 0 or 1
	\end{itemize}
\item and the number of items $n$ selected is small.
\end{itemize}
%----------------------------------------------------------------------------
\vspace{.5cm}
Then, in this case, making $n$ draws from the population doesn't change the probability of success much, and the hypergeometric r.v. can be approximated by a binomial r.v.

%----------------------------------------------------------------------------
\vspace{.5cm}
%\textbf{Rule of Thumb}

%----------------------------------------------------------------------------
\vspace{5cm}
%\hrule
%\vspace{.5cm}



%----------------------------------------------------------------------------
\begin{example}\ %\newline
Suppose a specific study region is known to have 2400 wolves,  of which 1100 have already been tagged. 

\begin{enumerate}
\item If 50 wolves are randomly captured, what is the probability that 20 of them have already been tagged?
\item Approximate the probability in part (1) using the binomial distribution.
\end{enumerate}
\textbf{Solution:}

\end{example}
%----------------------------------------------------------------------------
%\vspace{4cm}
%\hrule
%\vspace{.5cm}


%****************************************************************************************
\newpage

%----------------------------------------------------------------------------
}  % end large font
%----------------------------------------------------------------------------



\end{document}

