%\documentclass[12pt,letterpaper]{article}
\documentclass[12pt]{amsart}

\usepackage[left=1in, right=1in, top=.5in, bottom=.5in]{geometry}
\usepackage{latexsym,amssymb,amsmath,amsthm,amsopn,verbatim, mathpazo, graphicx, lmodern}
%\usepackage{latexsym,amssymb,amsmath,amsthm,amsopn,verbatim, mathpazo, graphicx}



\newcommand{\vs}{\vskip.5cm}
\newcommand{\hs}{\hskip1cm}
\newcommand{\ds}{\displaystyle}
\setlength{\parindent}{0pt}

\usepackage{hyperref}  % links, urls
\urlstyle{same}

\usepackage[T1]{fontenc}
\usepackage{libertine}
\renewcommand*\familydefault{\sfdefault}  %% Only if the base font of the document is to be sans serif


\newtheorem{theorem}{Theorem}[section]
\newtheorem{corollary}{Corollary}[theorem]
\newtheorem{lemma}[theorem]{Lemma}
\newtheorem{definition}[theorem]{Definition}
\newtheorem{example}[theorem]{Example}

\newcommand\indep{\protect\mathpalette{\protect\independenT}{\perp}}
\def\independenT#1#2{\mathrel{\rlap{$#1#2$}\mkern2mu{#1#2}}}

\newcommand\Pbb{\mathbb{P}}
\newcommand\Ebb{\mathbb{E}}
\newcommand\gl{\lambda}
\newcommand\pdfX{f_X(x)}
\newcommand\pdfY{f_Y(y)}
\newcommand\pdfXY{f_{X,Y}(x,y)}
\newcommand\cdfX{F_X(x)}
\newcommand\cdfY{F_Y(y)}
\newcommand\cdfXY{F_{X,Y}(x,y)}
\newcommand\pdfXgY{f_{X|Y}(x|y)}
\newcommand\pdfYgX{f_{Y|X}(y|x)}

\newcommand\intd{\displaystyle\int}
\newcommand\intinft{\int_{-\infty}^{\infty}}

\usepackage{fancyhdr}
\pagestyle{fancy}
\fancypagestyle{plain}{}
%\fancyhead{} % clear all header fields
%\fancyhf{}
\rhead{BSTA 550 Ch 29}   % from fancyhdr package
\renewcommand{\headrulewidth}{1pt}
\renewcommand{\footrulewidth}{0pt}



\begin{document}


%----------------------------------------------------------------------------
\setcounter{section}{29}
{\huge  
\section*{Chapter 29: Variance of \newline Continuous Random Variables}
}
%----------------------------------------------------------------------------

%----------------------------------------------------------------------------
{\large % begin large font
%----------------------------------------------------------------------------

%----------------------------------------------------------------------------
%\vspace{18cm}
%\hrule
%\vspace{1cm}



\vspace{.5cm}

\subsection{Expected value of a function of a continuous r.v.}\hspace*{\fill}%\newline

\vspace{.5cm}

%----------------------------------------------------------------------------
\textbf{How do we calculate the expected value of a function of a discrete r.v. or joint r.v.'s?}


%----------------------------------------------------------------------------
\vspace{5cm}
\hrule
\vspace{.5cm}



%----------------------------------------------------------------------------
\textbf{How do we calculate the expected value of a function of a continuous r.v. or joint r.v.'s?}


%----------------------------------------------------------------------------
\vspace{5cm}
\hrule
\vspace{.5cm}



%----------------------------------------------------------------------------
\begin{example}\label{29_EaX+b}\ %\newline
What is $\Ebb[aX+b]$?

\end{example}


%****************************************************************************************
\newpage
%----------------------------------------------------------------------------
\begin{example}\label{28_E_joint}\ \newline
Let $\pdfXY = 2e^{-(x+y)}$, for $0 \leq x \leq y$. Find $\Ebb[X]$.

\end{example}


\vfill
%----------------------------------------------------------------------------
\textbf{Remark}

If you are given $\pdfXY$ and want to calculate $\Ebb[X]$, you have two options:
\begin{enumerate}
\item Find $\pdfX$ and use it to calculate $\Ebb[X]$.
\item Or, calculate $\Ebb[X]$ using the joint density:
$$
\Ebb[X] = \intinft\intinft x \pdfXY dydx.
$$
\end{enumerate}

%----------------------------------------------------------------------------
%\vspace{2cm}
%\hrule
\vspace{2cm}



%****************************************************************************************
\newpage

%****************************************************************************************
\subsection{Important properties of expected values of functions of continuous r.v.'s}\hspace*{\fill}%\newline
%****************************************************************************************

\begin{enumerate}
\item $\Ebb[X+Y] = $

\vspace{5cm}

\item If $X_1, X_2, \ldots X_n$ are continuous r.v.'s and $a_1, a_2, \ldots a_n$ are constants, then $\Ebb[\sum_{i=1}^{n} a_i X_i] = $

\vspace{6cm}

\item If $X$ and $Y$ are independent continuous r.v.'s, and $g$ and $h$ are functions, then $\Ebb[g(X)h(Y)] = $


\vfill

\item If $X$ and $Y$ are independent continuous r.v.'s,  then \newline $\Ebb[XY] = $

\vspace{1cm}
\end{enumerate}

%****************************************************************************************
%****************************************************************************************
\newpage

%****************************************************************************************
\subsection{Variance of continuous r.v.'s}\hspace*{\fill}%\newline
%****************************************************************************************

\vspace{.5cm}


%----------------------------------------------------------------------------
\textbf{How do we calculate the variance of a discrete r.v.?}


%----------------------------------------------------------------------------
\vspace{5cm}
\hrule
\vspace{.5cm}



%----------------------------------------------------------------------------
\textbf{How do we calculate the variance of a continuous r.v.?}





%****************************************************************************************
\newpage

%----------------------------------------------------------------------------
\begin{example}\label{29_V_Uab}\ \newline
Let $\pdfX = \frac{1}{b-a}$, for $a \leq x \leq b$. Find $Var[X]$.

\end{example}

%----------------------------------------------------------------------------
\vspace{4cm}
\hrule
\vspace{.5cm}

%----------------------------------------------------------------------------
\begin{example}\label{29_V_Exp}\ \newline
Let $\pdfX = \gl e^{-\gl x}$, for $x > 0$ and $\gl > 0$. Find $Var[X]$.

\end{example}

%----------------------------------------------------------------------------
\vspace{4cm}
%\hrule
%\vspace{.5cm}


%****************************************************************************************
\subsection{Important properties of variances  of continuous r.v.'s}\hspace*{\fill}%\newline
%****************************************************************************************

\begin{enumerate}
\item $Var[aX+b] = $

\vfill

%\item $Var[X+Y] = $
%
%\vspace{6cm}


\item If $X_1, X_2, \ldots X_n$ are independent continuous r.v.'s and $a_1, a_2, \ldots a_n$ are constants, then $Var(\sum_{i=1}^{n} a_i X_i) = $


\vfill

\item If $X_1, X_2, \ldots X_n$ are independent continuous r.v.'s, then \newline $Var(\sum_{i=1}^{n} X_i) = $

\vfill
\end{enumerate}

%****************************************************************************************
\newpage


%----------------------------------------------------------------------------
\begin{example}\label{29_EV_cube}\ \newline
A machine manufactures cubes with a side length that varies uniformly from 1 to 2 inches. Assume the sides of the base and height are equal. The cost to make a cube is 10\textcent per cubic inch, and 5\textcent for the general cost per cube. Find the mean and standard deviation of the cost to make 10 cubes.



%%****************************************************************************************
%\newpage
%
%Solution to Example \ref{29_EV_cube} cont'd.

\end{example}

%----------------------------------------------------------------------------

%----------------------------------------------------------------------------
}  % end large font
%----------------------------------------------------------------------------



\end{document}

