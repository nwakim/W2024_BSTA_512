%\documentclass[12pt,letterpaper]{article}
\documentclass[12pt]{amsart}

\usepackage[left=1in, right=1in, top=.5in, bottom=.5in]{geometry}
\usepackage{latexsym,amssymb,amsmath,amsthm,amsopn,verbatim, mathpazo, graphicx, lmodern}
%\usepackage{latexsym,amssymb,amsmath,amsthm,amsopn,verbatim, mathpazo, graphicx}



\newcommand{\vs}{\vskip.5cm}
\newcommand{\hs}{\hskip1cm}
\newcommand{\ds}{\displaystyle}
\setlength{\parindent}{0pt}

\usepackage{hyperref}  % links, urls
\urlstyle{same}

\usepackage[T1]{fontenc}
\usepackage{libertine}
\renewcommand*\familydefault{\sfdefault}  %% Only if the base font of the document is to be sans serif


\newtheorem{theorem}{Theorem}[section]
\newtheorem{corollary}{Corollary}[theorem]
\newtheorem{lemma}[theorem]{Lemma}
\newtheorem{definition}[theorem]{Definition}
\newtheorem{example}[theorem]{Example}

\newcommand\indep{\protect\mathpalette{\protect\independenT}{\perp}}
\def\independenT#1#2{\mathrel{\rlap{$#1#2$}\mkern2mu{#1#2}}}

\newcommand\Pbb{\mathbb{P}}
\newcommand\Ebb{\mathbb{E}}
\newcommand\gs{\sigma}
\newcommand\gl{\lambda}
\newcommand\ga{\alpha}
\newcommand\gb{\beta}
\newcommand\pmfX{p_X(x)}
\newcommand\pmfY{p_Y(y)}
\newcommand\pdfX{f_X(x)}
\newcommand\pdfY{f_Y(y)}
\newcommand\pdfXY{f_{X,Y}(x,y)}
\newcommand\cdfX{F_X(x)}
\newcommand\cdfY{F_Y(y)}
\newcommand\cdfXY{F_{X,Y}(x,y)}
\newcommand\pdfXgY{f_{X|Y}(x|y)}
\newcommand\pdfYgX{f_{Y|X}(y|x)}
\newcommand\mgfX{M_X(t)}
\newcommand\mgfY{M_Y(t)}


\newcommand\intd{\displaystyle\int}
\newcommand\intinft{\int_{-\infty}^{\infty}}

\usepackage{fancyhdr}
\pagestyle{fancy}
\fancypagestyle{plain}{}
%\fancyhead{} % clear all header fields
%\fancyhf{}
\rhead{BSTA 550 Ch 36}   % from fancyhdr package
\renewcommand{\headrulewidth}{1pt}
\renewcommand{\footrulewidth}{0pt}



\begin{document}


%----------------------------------------------------------------------------
\setcounter{section}{36}
{\huge  
\section*{Chapter 36: Sums of Independent Normal Random Variables}
}
%----------------------------------------------------------------------------

%----------------------------------------------------------------------------
{\large % begin large font
%----------------------------------------------------------------------------

%----------------------------------------------------------------------------
%\vspace{18cm}
%\hrule
%\vspace{1cm}




%----------------------------------------------------------------------------
\begin{theorem} \ \  Let $X\sim N(\mu, \gs^2)$,  and let $Y=aX+b$, where $a$ and $b$ are constants. Then
$$
\hspace{-12cm} Y \sim 
$$
\end{theorem}

\begin{proof}
\end{proof}



%----------------------------------------------------------------------------
\vspace{5cm}
\hrule
\vspace{.5cm}

%----------------------------------------------------------------------------
\begin{theorem} \ \  Let $X_i \sim N(\mu_i, \gs_i^2)$ be independent normal rv's, for $i=1,2,\ldots,n$. Then %\newline 
$$
\hspace{-12cm} \sum_{i=1}^n X_i \sim 
$$
\end{theorem}

\begin{proof}
\end{proof}


%----------------------------------------------------------------------------
\vspace{.5cm}
\hrule
\vspace{.5cm}

\textbf{Special Cases}

\begin{enumerate}
\item Let $X_i \sim N(\mu, \gs^2)$ be iid normal rv's, for $i=1,2,\ldots,n$. Then
$$
\hspace{-12cm} \sum_{i=1}^n X_i \sim 
$$

\vspace{1cm}

\item Let $X_i \sim N(\mu, \gs^2)$ be iid normal rv's, for $i=1,2,\ldots,n$. Then
$$
\hspace{-12cm} \bar{X}=\frac{\sum_{i=1}^n X_i}{n} \sim 
$$

\vspace{1cm}

\item Let $X\sim N(\mu_X,\gs_X^2)$, and $Y\sim N(\mu_Y,\gs_Y^2)$. Then
$$
\hspace{-12cm} X-Y \sim 
$$

\begin{proof}
\end{proof}


\end{enumerate}
%----------------------------------------------------------------------------



%****************************************************************************************
\newpage

%----------------------------------------------------------------------------
\begin{example} \ \  Glaucoma is an eye disease that is manifested by high intraocular pressure (IOP). The distribution of IOP in the general population is approximately normal with mean 16 mmHg and standard deviation 3 mmHg.

\begin{enumerate}
\item Suppose a patient has 40 IOP readings. What is the probability that their average reading is greater than 20.32 mmHg, assuming their eyes are healthy?

\vspace{8cm}

\item Repeat the previous question for a patient with 10 IOP readings. 


\end{enumerate}
\end{example}



%----------------------------------------------------------------------------
}  % end large font
%----------------------------------------------------------------------------



\end{document}

