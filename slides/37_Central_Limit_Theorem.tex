%\documentclass[12pt,letterpaper]{article}
\documentclass[12pt]{amsart}

\usepackage[left=1in, right=1in, top=.5in, bottom=.5in]{geometry}
\usepackage{latexsym,amssymb,amsmath,amsthm,amsopn,verbatim, mathpazo, graphicx, lmodern}
%\usepackage{latexsym,amssymb,amsmath,amsthm,amsopn,verbatim, mathpazo, graphicx}



\newcommand{\vs}{\vskip.5cm}
\newcommand{\hs}{\hskip1cm}
\newcommand{\ds}{\displaystyle}
\setlength{\parindent}{0pt}

\usepackage{hyperref}  % links, urls
\urlstyle{same}

\usepackage[T1]{fontenc}
\usepackage{libertine}
\renewcommand*\familydefault{\sfdefault}  %% Only if the base font of the document is to be sans serif


\newtheorem{theorem}{Theorem}[section]
\newtheorem{corollary}{Corollary}[theorem]
\newtheorem{lemma}[theorem]{Lemma}
\newtheorem{definition}[theorem]{Definition}
\newtheorem{example}[theorem]{Example}

\newcommand\indep{\protect\mathpalette{\protect\independenT}{\perp}}
\def\independenT#1#2{\mathrel{\rlap{$#1#2$}\mkern2mu{#1#2}}}

\newcommand\Pbb{\mathbb{P}}
\newcommand\Ebb{\mathbb{E}}
\newcommand\gs{\sigma}
\newcommand\gl{\lambda}
\newcommand\ga{\alpha}
\newcommand\gb{\beta}
\newcommand\pmfX{p_X(x)}
\newcommand\pmfY{p_Y(y)}
\newcommand\pdfX{f_X(x)}
\newcommand\pdfY{f_Y(y)}
\newcommand\pdfXY{f_{X,Y}(x,y)}
\newcommand\cdfX{F_X(x)}
\newcommand\cdfY{F_Y(y)}
\newcommand\cdfXY{F_{X,Y}(x,y)}
\newcommand\pdfXgY{f_{X|Y}(x|y)}
\newcommand\pdfYgX{f_{Y|X}(y|x)}
\newcommand\mgfX{M_X(t)}
\newcommand\mgfY{M_Y(t)}


\newcommand\intd{\displaystyle\int}
\newcommand\intinft{\int_{-\infty}^{\infty}}

\usepackage{fancyhdr}
\pagestyle{fancy}
\fancypagestyle{plain}{}
%\fancyhead{} % clear all header fields
%\fancyhf{}
\rhead{BSTA 550 Ch 37}   % from fancyhdr package
\renewcommand{\headrulewidth}{1pt}
\renewcommand{\footrulewidth}{0pt}



\begin{document}


%----------------------------------------------------------------------------
\setcounter{section}{37}
{\huge  
\section*{Ch 37: The Central Limit Theorem (CLT)}
}
%----------------------------------------------------------------------------

%----------------------------------------------------------------------------
{\large % begin large font
%----------------------------------------------------------------------------

\vspace{.5cm}

%----------------------------------------------------------------------------
\begin{theorem}[Central Limit Theorem (CLT)] \ Let $X_i$ be iid rv's with common mean $\mu$ and variance $\gs^2$, for $i=1,2,\ldots,n$. Then
$$
\hspace{-12cm} \sum_{i=1}^n X_i \rightarrow
$$\end{theorem}

%----------------------------------------------------------------------------
\vspace{12cm}
\hrule
\vspace{.5cm}

%----------------------------------------------------------------------------
\begin{corollary} \ \  Let $X_i$ be iid rv's with common mean $\mu$ and variance $\gs^2$, for $i=1,2,\ldots,n$. Then
$$
\hspace{-12cm} \bar{X}=\frac{\sum_{i=1}^n X_i}{n}  \rightarrow
$$\end{corollary}





%****************************************************************************************
\newpage

%----------------------------------------------------------------------------
\begin{example} \ \  According to a large US study, the mean resting heart rate of adult women is about 74 beats per minutes (bpm), with standard deviation 13 bpm (NHANES 2003-2004).

\begin{enumerate}
\item Find the probability that the average resting heart rate for a random sample of 36 adult women is more than 3 bpm away from the mean. 

\vspace{12cm}

\item Repeat the previous question for a single adult woman. 


\end{enumerate}
\end{example}



%****************************************************************************************
\newpage


%----------------------------------------------------------------------------
\begin{example} \ \  Let $X_i \sim Exp(\gl)$ be iid r.v.'s for $i=1,2,\ldots,n$. Then
$$
\hspace{-12cm} \sum_{i=1}^n X_i \rightarrow
$$
\end{example}


%----------------------------------------------------------------------------
\vspace{5cm}
\hrule
\vspace{.5cm}



\textbf{CLT for Discrete rv's}

\begin{enumerate}
\item \textbf{Binomial rv's}: Let $X \sim Bin(n,p)$. 

\vspace{8cm}

\item \textbf{Poisson rv's}: Let $X \sim Poisson(\gl)$. 

\end{enumerate}
%----------------------------------------------------------------------------



%****************************************************************************************
\newpage

%----------------------------------------------------------------------------
\begin{example} \ \  Suppose that the probability of developing a specific type of breast cancer in women aged 40-49 is 0.001. Assume the occurrences of cancer are independent. Suppose you have data from a random sample of 20,000 women aged 40-49.

\begin{enumerate}
\item How many of the 20,000 women would you expect to develop this type of breast cancer, and what is the standard deviation?

\vspace{3cm}

\item Find the \textbf{exact} probability that more than 15 of the 20,000 women will develop this type of breast cancer.

\vspace{8cm}

\item Use the CLT to find the \textbf{approximate} probability that more than 15 of the 20,000 women will develop this type of breast cancer.

\vspace{5cm}

%****************************************************************************************
\newpage

\item Use the CLT to approximate the following probabilities, where $X$ is the number of women that will develop this type of breast cancer.
	\begin{enumerate}
	\item $\Pbb(15 \leq X \leq 22)$
%\vspace{3cm}
	\item $\Pbb(X > 20)$
%\vspace{3cm}
	\item $\Pbb(X < 20)$
%\vspace{3cm}
	\end{enumerate}

\vspace{8cm}


\item Find the \textbf{approximate} probability that more than 15 of the 20,000 women will develop this type of breast cancer - not using the CLT!

\vspace{7cm}

\item Use the CLT to approximate the approximate probability in the previous question!

\vspace{5cm}

\end{enumerate}
\end{example}



%----------------------------------------------------------------------------
}  % end large font
%----------------------------------------------------------------------------

\end{document}

