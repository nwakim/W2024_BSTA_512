%\documentclass[12pt,letterpaper]{article}
\documentclass[12pt]{amsart}

\usepackage[left=1in, right=1in, top=.5in, bottom=.5in]{geometry}
\usepackage{latexsym,amssymb,amsmath,amsthm,amsopn,verbatim, mathpazo, graphicx, lmodern}
%\usepackage{latexsym,amssymb,amsmath,amsthm,amsopn,verbatim, mathpazo, graphicx}



\newcommand{\vs}{\vskip.5cm}
\newcommand{\hs}{\hskip1cm}
\newcommand{\ds}{\displaystyle}
\setlength{\parindent}{0pt}

\usepackage{hyperref}  % links, urls
\urlstyle{same}

\usepackage[T1]{fontenc}
\usepackage{libertine}
\renewcommand*\familydefault{\sfdefault}  %% Only if the base font of the document is to be sans serif


\newtheorem{theorem}{Theorem}[section]
\newtheorem{corollary}{Corollary}[theorem]
\newtheorem{lemma}[theorem]{Lemma}
\newtheorem{definition}[theorem]{Definition}
\newtheorem{example}[theorem]{Example}

\newcommand\indep{\protect\mathpalette{\protect\independenT}{\perp}}
\def\independenT#1#2{\mathrel{\rlap{$#1#2$}\mkern2mu{#1#2}}}

\newcommand\Pbb{\mathbb{P}}
\newcommand\Ebb{\mathbb{E}}
\newcommand\gs{\sigma}
\newcommand\gl{\lambda}
\newcommand\ga{\alpha}
\newcommand\gb{\beta}
\newcommand\pmfX{p_X(x)}
\newcommand\pmfY{p_Y(y)}
\newcommand\pdfX{f_X(x)}
\newcommand\pdfY{f_Y(y)}
\newcommand\pdfXY{f_{X,Y}(x,y)}
\newcommand\cdfX{F_X(x)}
\newcommand\cdfY{F_Y(y)}
\newcommand\cdfXY{F_{X,Y}(x,y)}
\newcommand\pdfXgY{f_{X|Y}(x|y)}
\newcommand\pdfYgX{f_{Y|X}(y|x)}
\newcommand\mgfX{M_X(t)}
\newcommand\mgfY{M_Y(t)}


\newcommand\intd{\displaystyle\int}
\newcommand\intinft{\int_{-\infty}^{\infty}}

\usepackage{fancyhdr}
\pagestyle{fancy}
\fancypagestyle{plain}{}
%\fancyhead{} % clear all header fields
%\fancyhf{}
\rhead{BSTA 550 Ch 43 Part 2}   % from fancyhdr package
\renewcommand{\headrulewidth}{1pt}
\renewcommand{\footrulewidth}{0pt}



\begin{document}


%----------------------------------------------------------------------------
\setcounter{section}{43}
{\huge  
\section*{Chapter 43: Moment Generating Functions \newline
Part 2}
}
%----------------------------------------------------------------------------

%----------------------------------------------------------------------------
{\large % begin large font
%----------------------------------------------------------------------------

%----------------------------------------------------------------------------
%\vspace{18cm}
%\hrule
%\vspace{1cm}


\vspace{.5cm}
%----------------------------------------------------------------------------
\textbf{Recap: What is an mgf?} 



\vspace{5cm}

%----------------------------------------------------------------------------
\begin{example} \ \  Let $X$ be a random variable with mgf
$$
\mgfX = \frac{1}{5}e^t + \frac{3}{10}e^{2t} + \frac{1}{2}e^{3t}.
$$
Find the pmf or pdf of $X$.

\end{example}


%----------------------------------------------------------------------------
%\textbf{Where did Part 1 end?} 

%----------------------------------------------------------------------------
%\vspace{.5cm}

%****************************************************************************************
\newpage

%----------------------------------------------------------------------------
\begin{example} \ \  Let $X$ be a normal random variable with mean $\mu$ and variance $\gs^2$, i.e. $X \sim N(\mu,\gs^2)$. 

\begin{enumerate}
\item Find the mgf of $X$.

\vspace{6cm}

\item Find $\Ebb[X]$.

\vspace{4cm}

\item Find $Var(X)$.

\vspace{3cm}
\end{enumerate}

\end{example}

%----------------------------------------------------------------------------
\vspace{.5cm}
\hrule
\vspace{.5cm}


%----------------------------------------------------------------------------
\begin{theorem} \ \  Let $X$ have mgf $\mgfX$, and let $Y=aX+b$, where $a$ and $b$ are constants. Then
$$
\mgfY = 
$$
\end{theorem}

\begin{proof}
\end{proof}


%****************************************************************************************
\newpage

%----------------------------------------------------------------------------
\textbf{Question:}\textit{ Do linear transformations always preserve the distribution type?} \newline

\vspace{.2cm}
I.e., if $X$ has a certain probability distribution, does $aX+b$  always have the same distribution type?

%----------------------------------------------------------------------------
\vspace{.5cm}
\hrule
\vspace{.5cm}

%----------------------------------------------------------------------------
\begin{example} \ \  Let $X \sim U[0,1]$, and $Y = 2X+3$. \newline Is $Y$ also a uniform rv? If so, what are its parameters? 


\end{example}

\vfill
%\vspace{5cm}

%----------------------------------------------------------------------------
\begin{example} \ \  Let $X \sim Exp(\gl=5)$, and $Y = 2X+3$. \newline Is $Y$ also an exponential rv? If so, what is its parameter? 


\end{example}

\vfill
%----------------------------------------------------------------------------
%\vspace{5cm}
%\hrule
%\vspace{.5cm}

%****************************************************************************************
\newpage
%----------------------------------------------------------------------------
\textbf{Mgf's of Sums of Independent rv's} 

%----------------------------------------------------------------------------
\vspace{.5cm}
\hrule
\vspace{.5cm}

%----------------------------------------------------------------------------
\begin{theorem} \ \  Let $X_1, X_2, \ldots, X_n$ be independent rv's with respective mgf's $M_{X_i}(t)$, for $i=1,2,\ldots,n$. Let $Y=\sum_{i=1}^n a_iX_i$, where $a_i$ are constants. Then
$$
\mgfY = %\Pi_{i=1}^n M_{X_i}(a_it).
$$
\end{theorem}

\begin{proof}
\end{proof}

%%****************************************************************************************
%\newpage

%----------------------------------------------------------------------------
\vspace{6cm}
\hrule
\vspace{.5cm}

%----------------------------------------------------------------------------
\begin{example} \ \  Let $X_i \sim N(\mu_i, \gs_i^2)$ be independent normal rv's. \newline 
What is the distribution of\ \  $Y=\sum_{i=1}^n X_i$?
\end{example}


%----------------------------------------------------------------------------
\vspace{6cm}
\hrule
\vspace{.5cm}
%----------------------------------------------------------------------------
\begin{example} \ \  Let $X_i \sim N(\mu, \gs^2)$ be iid normal rv's, for $i=1,2,\ldots,n$. \newline 
 What is the distribution of\ \  $\bar{X}=\frac{\sum_{i=1}^n X_i}{n}$?

\end{example}

%****************************************************************************************
\newpage

%----------------------------------------------------------------------------
\begin{example} \ \  Let $Z$ be a standard normal random variable, i.e. $Z \sim N(0,1)$. \newline
Show that $Z^2 \sim \chi_1^2$, i.e. is a chi-squared rv with 1 degree of freedom.
\end{example}



%----------------------------------------------------------------------------
}  % end large font
%----------------------------------------------------------------------------



\end{document}

